\chapter{Krav}
\label{chap:krav}

Dette kapitel beskriver de funktionelle krav og ikke-funktionelle krav som systemet skal opfylde. Kravene er beskrevet med user-stories, og alle krav er prioriteret med MoSCoW.

\section{Funktionelle krav}

De funktionelle krav er beskrevet vha X antal epics, med dertilhørende user stories. Under hver user-story er kriterierne for accept beskrevet.

\subsection{Epic 1 - Semantisk søgning}

\textbf{Beskrivelse}: Implementer AI-drevet søgning der forstår betydningen af søgeforespørgsler og matcher med relevant indhold baseret på semantisk lighed frem for kun nøgleord.
\\
\\
\textbf{Bruger værdi}: Brugere kan finde relevant information selv når de ikke bruger præcise nøgleord, 
hvilket reducerer søgetid og forbedrer brugeroplevelsen.

\subsubsection{User Story 1.1}

Som \textbf{gæstebruger} vil jeg gerne søge efter information,
så jeg kan finde relevant indhold uden at logge ind
\\
\\
\textbf{Acceptance Criteria:}

\begin{itemize}
    \item Søgefunktion er tilgængelig uden login
    \item Resultater vises for alle brugergrupper (ikke personaliseret)
    \item Mindst 5 relevante resultater for almindelige søgninger
\end{itemize}

\subsubsection{User Story 1.2}

Som \textbf{content editor} vil jeg have at nyt indhold automatisk bliver søgbart,
så jeg ikke skal tænke på manuel re-indexering
\\
\\
\textbf{Acceptance Criteria:}
\begin{itemize}
    \item Når indhold publiceres, genereres embedding inden for 30 sekunder
    \item Når indhold opdateres, re-genereres embedding automatisk
    \item Fejl i indexering vises i backoffice
\end{itemize}


\subsection{Epic 2 - Personaliseret søgning}

\textbf{Beskrivelse}: Søgeresultater tilpasses automatisk baseret på brugerens rolle/gruppe,
så mest relevante indhold for deres arbejdsområde vises først.
\\
\\
\textbf{Bruger værdi}: Brugere slipper for at sortere gennem irrelevant indhold og finder hurtigere
de guider der matcher deres behov.

\subsubsection{User Story 2.2}
Som \textbf{bruger der er logget ind}, vil jeg gerne se søgeresultater der tilpasset min rolle, så jeg slipper for at gennemsøge irrelevant content.

\textbf{Acceptance Criteria:}
\begin{itemize}
    \item Systemet identificerer min brugergruppe automatisk
    \item Højt relevante resultater for min gruppe vises først
    \item Irrelevante resultater vises lavere eller skjules helt
    \item Visuelt indikator viser hvorfor resultat er relevant for mig
\end{itemize}

\subsection{Epic 3 - Semantisk svar}

\textbf{Beskrivelse:} Implementer semantisk svar på en søgning, så søgeresultatet også indeholder et sammenfattende svar på søge querien.
\\
\\
\textbf{Bruger værdi:} Brugeren får et potentielt hurtigere svar, og slipper for at dykke ned i for meget content.

\subsubsection{User Story 3.3}
Som \textbf{bruger} vil jeg gerne have et sammenfattende svar på min søgning, så jeg sparer tid på at gennemgå for meget indhold.
\\
\\
\textbf{Acceptance Criteria:}
\begin{itemize}
\item Et sammenfattende svar vises øverst i søgeresultaterne
\item Svaret er maksimalt 2-3 sætninger langt
\item Svaret inkluderer reference til de kilder der blev brugt
\item Hvis der ikke kan genereres et sikkert svar, vises kun almindelige søgeresultater
\end{itemize}

% \begin{table}[H]
% \centering
% \begin{tabular}{@{}lp{10cm}@{}}
% \toprule
% \textbf{ID} & \textbf{Krav} \\ \midrule
% R1 & Beskrivelse af funktionelt krav 1 \\
% FR2 & Beskrivelse af funktionelt krav 2 \\
% FR3 & Beskrivelse af funktionelt krav 3 \\
% \bottomrule
% \end{tabular}
% \caption{Funktionelle krav}
% \label{tab:funktionelle_krav}
% \end{table}

\section{Ikke-funktionelle krav}
% \begin{table}[H]
% \centering
% \begin{tabular}{@{}lp{10cm}@{}}
% \toprule
% \textbf{ID} & \textbf{Krav} \\ \midrule
% NFR1 & Beskrivelse af ikke-funktionelt krav 1 \\
% NFR2 & Beskrivelse af ikke-funktionelt krav 2 \\
% NFR3 & Beskrivelse af ikke-funktionelt krav 3 \\
% \bottomrule
% \end{tabular}
% \caption{Ikke-funktionelle krav}
% \label{tab:ikke_funktionelle_krav}
% \end{table}
